\documentclass{article}
\usepackage{graphicx}
\usepackage{amsmath}
\usepackage{tikz}
\usetikzlibrary{arrows.meta}
\usepackage{listings}
\usepackage{xcolor}
\usepackage{caption}


\lstdefinestyle{wsl}{
  backgroundcolor=\color{black},
  basicstyle=\color{white}\ttfamily,                   
  numbers=left,
  numbersep=5pt,                                        
  frame=single,            
  rulecolor=\color{gray},         
  framexleftmargin=15pt,          
  numberstyle=\tiny\color{gray}   
}


\title{Simulazione della dinamica del moto di un punto materiale in un biliardo triangolare}
\author{Cesare Sabattini\\ Mattia Tonti}
\date{18 May 2024}

\begin{document}

\maketitle

\section*{Abstract}
Il programma è volto alla simulazione di collisioni multiple e sequenziali di un punto materiale all'interno 
di un biliardo triangolare, con lo specifico obiettivo di studiare la distribuzione delle ordinate Y e degli angoli di uscita \(\theta_f\), in funzione di \(y_i\) e \(\theta_i\), dati i parametri strutturali del biliardo.
Il source code è reperibile all'interno della repository GitHub presso git@github.com:CesareSabattini/Triangular\_Billiard.git

\section{Introduzione}
L'applicazione si suddivide in due sezioni pricipali: simulazione della traiettoria, con supporto grafico e possibilità di inserimento di dati strutturali del biliardo,  \(y_i\) e \(\theta_i\), ed analisi statistica dei dati raccolti, a partire da distribuzioni gaussiane delle variabili di ingresso.
Il tutto è implementato interamente secondo lo standard ISO C++20, con l'ausilio del framework SFML per la realizzazione della GUI.

\section{Analisi del problema}
\subsection{Simulazione}
Il sistema è composto da un trapezoide isoscele,e da un punto materiale interno ad esso, in moto. Le collisioni con le pareti del biliardo sono modellizzate come urti elastici.
Dalle leggi della riflessione si possono derivare le equazioni della traiettoria, con semplici analisi geometriche.


\begin{figure}[ht]
    \centering
\begin{tikzpicture}

\draw[thick] (-0.2,2) -- (8,-3);
\draw[dotted] (-1,0) -- (7,0);
\draw[dotted] (-3,-2) -- (3,-2);
\draw (-1,-2) -- (3,0);
\draw[->] (3,0) -- (2,-2);
\draw[dotted] (3,0) -- (1,-1.9);

\draw[thick] (2,0) arc[start angle=180,end angle=145,radius=1];
\draw[thick] (0,-2) arc[start angle=0,end angle=27,radius=1];
\draw[thick] (4,0) arc[start angle=0,end angle=-118,radius=1];
\draw[thick] (3.9,0) arc[start angle=0,end angle=-118,radius=0.9];
\draw[thick] (3.9,0) arc[start angle=0,end angle=-118,radius=0.9];
\draw[thick] (2.2,-1.5) arc[start angle=-120,end angle=-150,radius=1];
\draw[thick] (1.7,-1.3) arc[start angle=-101,end angle=-153,radius=0.7];

\node at (1.4,0.4) {$\alpha$};
\node at (0.3,-1.7) {$\theta_i$};
\fill[white] (4.1,-1.1) rectangle (4.5,-0.9);
\node[fill=white] at (4.2, -1) {$\theta_{i+1}$};
\node at (1.1,-1.4) {$\phi_i$};
\node at (1.8,-1.6) {$\phi_{i+1}$};

\end{tikzpicture}
    \caption{
    \centering
    Schema rappresentativo di un urto elastico contro la parete superiore del biliardo, che mostra le convenzioni applicate per il calcolo degli angoli.}
    \label{fig:wavefront}

\end{figure}

Definito \(\alpha\)l'angolo di inclinazione delle pareti
\footnote{Ai fini del calcolo si considerano gli angoli calcolati rispetto all'orizzontale,a meno di ulteriore indicazione.}

\begin{equation}
\alpha=\arctan{(\frac{R_1-R_2)}{L}}\\
\end{equation}

a seguito dell'urto i-esimo, per i$>$1, vale
\footnote{Viene considerato come urto 0, il lanco del punto materiale}
\begin{equation}
\centering
\theta_{i+1}=-(|\theta_{i}|+2\alpha) 
\end{equation}

Per i=1 vale\\
\begin{equation}
\left\{
\begin{aligned}
    |\theta_1|=2\alpha-|\theta_0| \hspace{2.5 mm} se\hspace{2.5 mm}|r_2|<|y|<|r_1|,\hspace{2.5 mm} \theta_0<\alpha ,\hspace{2.5 mm}  y_0\theta_0<0  \\
    \theta_1=-(|\theta_i|+2\alpha) \hspace{2.0mm} altrimenti
\end{aligned}
\right.
\end{equation}\\

Per il calcolo della collisione i+1 è sufficiente risolvere il sistema lineare:

\begin{equation}
\left\{
\begin{aligned}
    y_{i+1}=y_i+\tan\theta_i \cdot (x_{i+1}-x_i) \\
    |y_{i+1}|= R_1 - \tan\alpha \cdot x_{i+1}
\\
\end{aligned}
\right.
\end{equation}

\subsection{Analisi statistica}

Si analizzano le distribuzioni delle ordinate e degli angoli di uscita, esito di n simulazioni indipendenti, nelle variabili y e \(\theta\) di partenza, dati i parametri strutturali del biliardo e si studia la dipendenza della distribuzione da questi ultimi.
Y\_i e \(\theta\)\_i sono generati secondo distribuzioni gaussiane, con media \(\mu\) e deviazione standard \(\sigma\) dinamicamente programmabili:

\begin{equation}
    f(x) = \frac{1}{\sigma\sqrt{2\pi}} e^{-\frac{(x-\mu)^2}{2\sigma^2}}
\end{equation}

Dopo aver simulato l'evoluzione del sistema, si analizzano le distribuzioni e gli angoli di uscita dei punti materiali che escono dal biliardo, e se ne calcolano media

\begin{equation}
    \mu = \frac{1}{N} \sum_{i=1}^N x_i
\end{equation}

deviazione standard della media:
\begin{equation}
    \sigma_m = \frac{\sigma}{\sqrt{N}} = \frac{\sqrt{\frac{1}{N-1} \sum_{i=1}^N (x_i - \mu)^2}}{\sqrt{N}}
\end{equation}



coefficiente di simmetria
\footnote{calcolato con il coefficiente di Fisher-Pearson ottimizzato}

\begin{equation}
    \gamma_1 = \frac{\sqrt{N\cdot (N-1)}}{N(N-2)} \sum_{i=1}^N \left(\frac{x_i - \bar{x}}{s}\right)^3
\end{equation}

e coefficiente di appiattimento:

\begin{equation}
   \gamma_2 =  \sum_{i=1}^N \left(\frac{x_i - \bar{x}}{s}\right)^4 - 3
\end{equation}

\section{Struttura del codice}
\subsection{Simulation}
Nel namespace simulation, e nei relativi annidati, sono implementate le classi template, restrittive secondo il concept DoubleOrFloat, Ball, Pool, Collision e System.
L'interfaccia system amministra l'apparato di simulazione, e memorizza i dati delle collisioni in un array. Il metodo System::throwTheBall() computa la prima collisione, mentre quelle successive sono calcolate da System::computeNextCollision(). System::computeOutputY() computa l'ordinata di uscita in base ai dati dell'utlima collisione avvenuta.

\subsection{Analysis}

La classe Analyzer contiene i metodi necessari per lo studio della distribuzione multivariata delle ordinate di uscita, in funzione delle ordinate e degli angoli di ingresso, dati i parametri strutturali del biliardo.
Il metodo generate, utilizzando std::random\_device e std::mt19937, ossia un generatore di numeri casuali provvisto dalla Standard Library, basato sull'algoritmo Marsenne Twister, inserisce campioni di distribuzioni gaussiane all'interno di un vettore di array, input.
I parametri delle gaussiane e il numero di simulazioni sono dinamicamente inseribili dalla UI, e sono contenuti nella struct Params.
Sono stati inseriti vincoli nella generazione, per evitare di eccedere le dimensioni del biliardo e il dominio degli angoli.
Il metodo Analyzer::simulate() genera le simulazioni mediante System::simulate(), con parametri di volta in volta aggiornati, secondo il contenuto di input.
Il metodo Analyzer::analyze() implementa la computazione dei risultati relativi alle distribuzioni esito, tramite metodi di classe, e il loro inserimento in membro di tipo Result.

\subsection{Graphics}

Il namespace graphics contiene la logica di implementazione della GUI mediante SFML.
La grafica è organizzata nei namespaces scenes e components ed amministrata dalla classe MainWindow.
Le scene rappresentano le pagine della UI, mentre le componenti sono elementi comuni alle scenes, come input testuali o bottoni.
Ogni scene presenta tre metodi in comune: initializeComponents(), draw() e processEvents(), secondo la canonica procedura di inizializzazione di elementi di SFML.


La classe MainWindow amministra la logica del programma, tanto che il suo metodo run() è l'unico ad essere invocato nel main() dell'applicazione.
La grafica è modularizzata in finestre (Menu, ConfigSimulation, SimulationWindow, AnalysisWindow ed ErrorPopup), che vengono visualizzate a seconda del valore della selectedScene, ovvero un'istanza condivisa della enum class Scene, dichiarata nella MainWindow e condivisa con le componenti.
Tutti i colori utilizzati sono interni alla classe Colors, e il font è inizializzato unicamente nel costruttore di AppStyle, in MainWindow, e condiviso come shared\_ptr alle componenti.
In un folder components sono contenute tutte le classi relative ad elementi condivisi da più finestre.
L'error handling è gestito mediante eccezioni, che restituiscono un popup di errore.


\subsection{Gestione delle Risorse}
Le risorse sono organizzate al fine di minimizzare operazioni dispendiose non necessarie.
Nello specifico, si è fatto ampio uso di std::shared\_ptr per la gestione di oggetti di tipo System, al fine di mantenere la stessa istanza condivisa nella grafica e nel testing.
Analogamente si è optato per il font, evitando uploading multipli.
Le operazioni di copia sono minime, e limitate a tipi primitivi del linguaggio.
E' stata inoltre prestata attenzione alla const correctness.
Questo ha permesso di organizzare una struttura centralizzata attorno alla componente MainWindow, che possa amministrare grafica e apparato simulativo coerentemente ed efficacemente.

\subsection{Error Handling}
Si è optato per le eccezioni come meccanismo principale di runtime error handling. Le eccezioni sono lanciate in diversi punti del codice, come nei costruttori delle componenti della simulazione, per garantire la correttezza della loro inizializzazione.
I blocchi try catch sono limitati dunque alle sezioni di codice soggette a possibili throw: nella grafica, per esempio, un blocco catch intercetta le eccezioni scaturite internamente alle scenes, ed esegue il popup di una finestra di errore contenente il messaggio relativo.



\subsection{Testing}

Il programma provvede due macro di testing, implementate con Doctest, per simulazione ed analisi.
Il metodo System::Simulate(), presentando numerosi loggers, avrebbe potuto interferire con l'interfaccia di visualizzazione dei risultati del testing, motivo per cui è stata implementata una mock class, che erediti da System e che ne ridefinisca il metodo, eliminando loggers superflui.

\subsection{Compilazione}
Come building tool è stato utilizzato CMake: lo standard di linguaggio è C++20.
Sono attivate le flags di di rilevazione di warnings e memory leaks al runtime con Address Sanitizer.

Per la configurazione del progetto su LinuxOS e WSL si raccomanda di seguire i seguenti passaggi:

\begin{itemize}
    \item Installare le dipendenze necessarie:
    \begin{lstlisting}[style=wsl]
sudo apt update
sudo apt install build-essential cmake libsfml-dev
\end{lstlisting}

\item Creare una directory di build:
    \begin{lstlisting}[style=wsl]
mkdir build
cd build
\end{lstlisting}

\item Configurare la soluzione con CMake:
    \begin{lstlisting}[style=wsl]
cmake ..
\end{lstlisting}

\item Compilare la soluzione:
    \begin{lstlisting}[style=wsl]
cmake --build .
\end{lstlisting}


\item Per avviare l'eseguibile dalla directory build:
    \begin{lstlisting}[style=wsl]
./Triangular_Pool
\end{lstlisting}

\item Per eseguire il testing:
    \begin{lstlisting}[style=wsl]
./tests
\end{lstlisting}

\end{itemize}

\section{Bibliografia e sitografia}

Author
Title of the source
Year of publication
Place of publication
Edition
Publisher
Editor
(URL and date of retrieval for internet sources)
\begin{itemize}
    \item Bjarne Stroustrup, "C++. Linguaggio, libreria standard, principi di programmazione", 2019, Pearson.
    \item SFML 2.6, https://www.sfml-dev.org/tutorials/2.6/
    \item Microsoft Learn, https://learn.microsoft.com/it-it/cpp/?view=msvc-170, per i seguenti temi:
    \begin{itemize}
        \item Generazione di numeri casuali (https://learn.microsoft.com/it-it/cpp/standard-library/random?view=msvc-170).
        \item Implementazione di template classes tra .hpp e .cpp:
        https://learn.microsoft.com/en-us/cpp/cpp/source-code-organization-cpp-templates?view=msvc-170
    \end{itemize}
      \item Per le definizioni utilizzate di coefficienti di simmetria ed appiattimento, NIST, https://www.nist.gov/

      \item DOCtest 2.4.11, https://github.com/doctest/doctest
      \item cppreference, https://en.cppreference.com/w/, prevalentemente per la Standard Library e concepts.
      \item Angular convectional commit, https://www.conventionalcommits.org/en/v1.0.0-beta.4/   
\end{itemize}

\end{document}

